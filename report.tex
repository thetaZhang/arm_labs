\documentclass[12pt, a4paper, oneside]{ctexart}
\usepackage{fancyhdr}
\usepackage{amsmath, amsthm, amssymb, bm, graphicx, hyperref, mathrsfs, graphicx, float, subfigure, caption, makecell, longtable,framed,booktabs}
\usepackage[dvipsnames]{xcolor}
\usepackage{listings}
\usepackage{indentfirst}
\usepackage{labels}
\usepackage{enumitem}
\setlength{\parindent}{2em}
\renewcommand{\lstlistingname}{code}
\lstset{
    language=C++, % 设置语言
 basicstyle=\ttfamily, % 设置字体族
 breaklines=true, % 自动换行
 keywordstyle=\bfseries\color{NavyBlue}, % 设置关键字为粗体,颜色为 NavyBlue
 morekeywords={}, % 设置更多的关键字,用逗号分隔
 emph={self,input,output,wire,reg,posedge,negedge}, % 指定强调词,如果有多个,用逗号隔开
    emphstyle=\bfseries\color{Rhodamine}, % 强调词样式设置
    commentstyle=\itshape\color{black!50!white}, % 设置注释样式,斜体,浅灰色
    stringstyle=\bfseries\color{PineGreen!90!black}, % 设置字符串样式
    columns=flexible,
    numbers=left, % 显示行号在左边
    numbersep=2em, % 设置行号的具体位置
    numberstyle=\footnotesize, % 缩小行号
    frame=single, % 边框
    framesep=1em % 设置代码与边框的距离
}
\usepackage[left=1in, right=1in, top=1in, bottom=1in]{geometry}


\ctexset{
    % 修改 section。
    section={   
        format=\heiti\raggedright\zihao{-2} % 设置 section 标题为黑体、右对齐、小4号字
    },
    % 修改 subsection。
    subsection={   
        format=\heiti\zihao{4} % 设置 subsection 标题为黑体、5号字
    }
}


\pagestyle{fancy}
\fancyhf{}
\renewcommand{\headrulewidth}{0pt}
\fancyfoot[C]{\thepage}

\title{\textbf{ARM实验报告(讨论题)}}
\author{张浩宇 522031910129}
\date{}

\begin{document}
    \maketitle
    \section{实验一:时钟选择与 GPIO 实验}
    \begin{enumerate}[listparindent=2em]
        \item 在硬件不改变的情况下,在程序中将内部时钟
        从16M改为8M,或将外部时钟从25M改为30M,会有什么结果?
        
        {\kaishu 由于时钟频率的改变,使系统频率变化,会导致延时函数的延时时间的改变,从而使闪烁频率改变。}

        \item 能否将启用PLL后的系统时钟频率调整到
        PIOSC(16M)或MOSC(25M)以下?
        
        {\kaishu 可以。将PLL频率配置为300MHz或400,可输出此PPL频率的任意分频作为
        系统频率,最低可配置为6Mhz,因此可以将系统时钟调整到PIOSC(16M)或MOSC(25M)以下。}

        
        \item 将PLL后的时钟调整为最大值 120M,LED 闪烁会有什么变化?为什么?
        
        {\kaishu 会使LED闪烁频率很快,在快闪时由于闪烁变化超过人眼的分辨能力,会表现为常亮。}

        \item 理解\verb|GPIOPinWrite(GPIO_PORTF_BASE, GPIO_PIN_0, GPIO_PIN_0)|函数中每个函数
        参数的意义。第三个函数项为\verb|GPIO_PIN_0|,如果改为 1 或改为 2,或其他值,分别有什么
        现象?

        {\kaishu 第一个参数表示操作的GPIO端口基地址,这里表示操作端口F;第二个参数表示操作的引脚,这里仅操作第一个引脚PF0;
        第三个参数表示操作的引脚的电平值,这里仅将PF0设置为高电平,PF0的LED点亮。

        \verb|GPIO_PIN_0|的值得为0x01。如果改为1,PF0引脚的状态仍为高电平,现象不变。
        如果改为2,即0x02,PF0引脚处为低电平,LED将熄灭。
        如果修改为其他值,仅当最低位即PF0对应位置为1时,PF0的LED点亮,否则将熄灭
        }

        \item 结合硬件说明\verb|GPIOPinConfigure|行的作用。如果此行注释,在 WATCH 窗口中观察
        \verb|key_value|值会有什么变化?

        {\kaishu 该行的作用是GPIO引脚配置,将PJ0和PJ1输入配置为驱动强度2mA、推挽上拉模式。
        如果此行注释,}

        \item 为什么在慢闪时,按住 \verb|USR_SW1| 时,没有马上进入快闪模式?
        
        {\kaishu 在实验一的程序中,闪烁是通过阻塞延时来实现的,其在延时时,MCU被完全阻塞,
        不执行其他任务。若在阻塞延时中按下开关,则必须等当前延时结束后才会处理响应按键按下,
        因此不能马上进入快闪模式。}

       
        
       
       

    \end{enumerate}

    \section{实验二:I2C GPIO 扩展及 SYSTICK 中断实验}
    \begin{enumerate}[listparindent=2em]
        \item  在使用PIOSC 及 MOSC 时,能否生成频率不等于晶振频率的系统时钟?如内部晶振为
        16M,能否生成10M的系统时钟频率。

        {\kaishu 可以。在使用PIOSC 及 MOSC 时,
        可以通过PLL来生成频率不等于晶振频率的系统时钟。
        例如,要生成10M的系统时钟频率,只需在初始化时调用下列配置函数:
        \begin{lstlisting}
SysCtlClockFreqSet((SYSCTL_OSC_INT|SYSCTL_USE_PLL|SYSCTL_CFG_VCO_480),10000000);
        \end{lstlisting}
        }

        \item 在使用PLL 时,系统频率最小值及最大值分别为多少?
        
        {\kaishu 在使用PLL 时,系统频率最小值为6Mhz,最大值为120Mhz。}
        
        \item 如果跑马灯要求为2位跑马,例:当显示为 1 时,跑马灯点亮 LED8,LED1,当显
        示为 2 时,跑马灯点亮 LED1,LED2,如此循环,如何实现?

        {\kaishu 定义一个\verb|uint8_t|类型的变量\verb|bit_choose|作为数码管的位选,初始值为0x81,
        当SysTick计数达到上限,输出标志,要改变数码管状态时,按照如下方式改变\verb|bit_choose|的值:}

        \begin{lstlisting}
        if (value & 0x80) {  // 最高位的1移到最低位
            value = (value << 1) | 0x01;  
        } else {
            value <<= 1; 
        }
        \end{lstlisting}

        {这样每次位选两个数码管并且移动,实现2位跑马。}

        \item 在 3 基础上,数码管显示也改为 2 位显示。例
        
        第一步 显示 1,2,跑马灯显示 LED1,2
        
        第二步 显示 2,3,跑马灯显示 LED2,3
        
        ......
        
        第八步 显示 8,1,跑马灯显示 LED8,1
        
        第九步 回到第一步

        {\kaishu 在3的基础上,要实现数码管不同位显示不同的数字,需要对数码管快速的扫描,快速切换位选和断码,
        以表现出显示不同的数字。
        定义一个\verb|uint8_t|类型的变量\verb|bit_choose|作为数码管的位选,指向每次点亮的两个数码管的较前一个,初始值0x01,
        定义一个\verb|uint8_t|类型的变量\verb|seg_display|表示当前显示的数字段码索引,初始值为0。

        在SysTick中断服务函数中,设定100ms和1ms的两个计数标志,分别作为数码管扫描和数码管滚动显示的标志。}
        }
        \begin{lstlisting}
void SysTick_Handler(void)
{
    if (systick_100ms_couter	!= 0)
	    systick_100ms_couter--;
    else
    {
	    systick_100ms_couter	= SYSTICK_FREQUENCY/10;
	    systick_100ms_status 	= 1;
    }
    if (systick_1ms_couter	!= 0)
        systick_1ms_couter--;
    else
    {
        systick_1ms_couter	= SYSTICK_FREQUENCY/1000;
        systick_1ms_status 	= 1;
    }
}   
        \end{lstlisting}

        {\kaihsu 每当\verb|systick_100ms_status|为1时,切换位选和段码,以显示不同的数字,
        每当\verb|systick_1ms_status|为1时,进行扫描显示。} 
        \begin{lstlisting}
void main(){
    ...... 
    while(1) {
        ...... 
        if (systick_100ms_status == 1) {
            systick_100ms_status = 0;
            bit_choose = (bit_choose << 1) | (bit_choose >> 7) // 位选移动
            seg_display = (seg_display + 1) % 8; // 数码管显示索引移动
        }

        if (systick_1ms_status == 1) {
            systick_1ms_status = 0;
            Display( bit_choose, seg_display); 
        }
    }
}
        \end{lstlisting}

    {\kaishu \verb|Display(uint8_t bit_choose, uint8_t seg_display)|函数实现一次显示扫描。}

    \begin{lstlisting}
void Display(uint8_t bit_choose, uint8_t seg_display) {
    static uint8_t extra_bit=0; // 标志当前显示第一位还是第二位
    result = I2C0_WriteByte(TCA6424_I2CADDR,TCA6424_OUTPUT_PORT1,seg7[(seg_display+extra_bit)%8]); // 段选	
    result = I2C0_WriteByte(TCA6424_I2CADDR,TCA6424_OUTPUT_PORT2,(bit_choose << extra_bit) | (bit_choose >> 7*extra_bit )); // 位选
    extra_bit!=extra_bit; 
}
    \end{lstlisting}

        \item 用 \verb|USR_SW1| 控制跑马灯的频率,
        
        按第 1 下,间隔为 1S;
        
        按第 2 下,间隔为 2S;

        按第 3 下,间隔为 0.2S;
        
        按第 4 下,回到上电初始状态,间隔 0.5S。
        
        以 4 为模,循环往复。

        {\kaishu 在SysTick中断服务函数中,检测按键的按下,根据按键的按下次数,改变跑马灯计数上限标志的值。}

        \begin{lstlisting}
void SysTick_Handler(void)
{
    static uint8_t key_value, key_value_last=1,key_press_count=0;
    if (systick_couter	!= 0)
	    systick_couter--;
    else
    {
	    systick_couter	= SYSTICK_MAX; // 达到上限SYSTICK_MAX时触发标志
	    systick_status 	= 1;
    }

    key_value = GPIOPinRead(GPIO_PORTJ_BASE, GPIO_PIN_0); // 读取按键状态
    if (key_value == 0 && key_value_last != 0) { // 按键按下时
        key_press_count++;
        if (key_press_count == 1) {
            SYSTICK_MAX=SYSTICK_FREQUENCY; // 1s
        } else if (key_press_count == 2) {
            SYSTICK_MAX=2*SYSTICK_FREQUENCY; // 2s
        } else if (key_press_count == 3) {
            SYSTICK_MAX=SYSTICK_FREQUENCY/5; // 0.2s
        } else if (key_press_count == 4) {
            SYSTICK_MAX=SYSTICK_FREQUENCY/2; // 0.5s
            key_press_count = 0; // 回到初始状态
        }
    } 

    key_value_las=key_value;
}
        \end{lstlisting}

        \item 请编程在数码管上实现时钟功能,在数码管上最左端显示分钟+秒数,其中分钟及秒数均为 2 位数字。
        如 12:00,共 5 位。
        
        每隔一秒,自动加 1,当秒数到 60 时,自动分钟加 1,秒数回到 00,分钟及秒数显示范围 00~59。
        
        当按下 \verb|USR_SW1| 时,秒数自动加 1;
        
        当按下 \verb|USR_SW2| 时,分钟自动加 1;
        
        当按下以上一个或两个按键不松开时,对应的显示跳变数每隔 200mS 自动加 1。即如下

        按下 \verb|USR_SW1| 1S,则显示跳变秒数加 5。

        {\kaishu 定义全局变量\verb|min|和\verb|sec|存储当前时间,Systick的1s计时标志有效时,增加对应时间值,并通过快速扫描实现
        时间在数码管上显示。}
        \begin{lstlisting}
void main(){
    ...... 
    while(1) {
        ...... 
        if (systick_1s_status == 1) { // 时间增加和进位
            systick_1s_status = 0;
            sec++;
            if (sec == 60) {
                sec = 0;
                min++;
                if (min == 60) {
                    min = 0;
                }
            }
        if (systick_1ms_status == 1) { // 扫描显示时钟
            systick_1ms_status = 0;
            Display();
        }
        
    }
}
        \end{lstlisting}

    {\kaishu 其中\verb|Display()|函数实现数码管的快速扫描显示。}

    \begin{lstlisting}
void Display() {
    static uint8_t i=0; // 指示本次扫描的位置
    // 构造当前显示的段码
	uint8_t seg_array[8];
	seg_array[0]=(seg7[(min)/10];
	seg_array[1]=(seg7[(min)%10];
	seg_array[2]=0x40;
	seg_array[3]=(seg7[(sec)/10];
	seg_array[4]=(seg7[(sec)%10];
    
    // 位选和段选显示本次扫描位的数字
	result = I2C0_WriteByte(TCA6424_I2CADDR,TCA6424_OUTPUT_PORT1,seg_array[i]);										
	result = I2C0_WriteByte(TCA6424_I2CADDR,TCA6424_OUTPUT_PORT2,(uint8_t)(0x01 << i));
	i++;
	if (i==5) i=0; // 仅用前5位显示
}
    \end{lstlisting}

    {\kaishu 在SysTick中断服务函数中,根据按键按下情况,改变当前计时变量值和1s计数上限}
\begin{lstlisting}
void SysTick_Handler(void)
{
    static uint8_t key_value1, key_value1_last=1,key_value2, key_value2_last=1;
    key_value1 = GPIOPinRead(GPIO_PORTJ_BASE, GPIO_PIN_0); // 读取按键状态
    key_value2 = GPIOPinRead(GPIO_PORTJ_BASE, GPIO_PIN_1); // 读取按键状态
    if (key_value == 0 && key_value_last != 0) { // 按键1按下时
        SYSTICK_1s_MAX=SYSTICK_FREQUENCY; // 1s
        sec++;
        if (sec == 60) {
            sec = 0;
            min++;
            if (min == 60) {
                min = 0;
            }
        }
    } 
    else if(key_value == 0 && key_value_last == 0) { // 按键1长按时
        SYSTICK_1s_MAX=SYSTICK_FREQUENCY/5; // 0.2s // 改变计数上限
    }
    else {
        SYSTICK_1s_MAX=SYSTICK_FREQUENCY; // 1s
    }
    if (key_value2 == 0 && key_value2_last != 0) { // 按键2按下时
        min++;
        if (min == 60) {
            min = 0;
        }
    }
    
    if (systick_1s_couter	!= 0) // 计时增加计数
	    systick_1s_couter--;
    else
    {
	    systick_1s_couter	= SYSTICK_1s_MAX; // 达到上限SYSTICK_MAX时触发标志
	    systick_1s_status 	= 1;
    }
    if (systick_1ms_status == 1) { // 扫描显示计数
        systick_1ms_status = 0;
        Display( bit_choose, seg_display); 
    }

   

    key_value_las=key_value;
}
\end{lstlisting}
    \end{enumerate}
    
    \section{实验三:UART 串行通讯口及中断优先级实验}

    \begin{enumerate}[listparindent=2em]
        \item 实验 3\-2,\verb|if (UARTCharsAvail(UART0_BASE))|此行程序的作用。如果没有此行,会导致
        什么问题?

        {\kaishu 此行程序的作用是判断UART0接收FIFO中是否有数据,如果有数据时接收并按原样发送,没有数据则跳过。
        
        如果没有此行,每次主程序循环串口会一直进行阻塞接收和发送,程序会阻塞在此,不能执行跑马灯等其他任务。}

        \item 实验 3\-3,\verb|void UART0_Handler(void)|为什么没有在主函数声明?
        
        {\kaishu \verb|void UART0_Handler(void)|是一个中断服务函数,
        其函数名是固定的,在\verb|startup_TM4C129.s|文件中已默认注册,
        在主函数中重写即可覆盖该函数,无需另声明。}
        
        \item 为什么 3\-3 的中断中需要读取中断标志并清除,而 \verb|SYSTICK| 不需要?
        
        {\kaishu 因为SysTick中断会被硬件自动清除。}

        \item 请根据上位机发出的命令字符串,如“MAY+01”,其中:
        
        MAY 为月份,(JAN,FEB,…DEC)均为三位。

        “+”表示加运算符,“-”表示减运算符,均为 1 位。

        01 表示增加或减少量,均为 2 位。范围 00-11。

        以上均为 ASCII 码,MAY+01 应该回之以JUNE,MAY-06 应该回之以 NOV。

        {\kaishu 在UART0中断服务函数中,对接收的字符串判断,并根据不同情况进行处理,
        发送相应的字符串即可。
        
        定义一个表示月份的全局数组以便操作,并定义两个全局变量分别表示接收到的月份和要增加的月份。}
        
    \begin{lstlisting}     
char *month[] = {"JAN", "FEB", "MAR", "APR", "MAY", "JUN", "JUL", "AUG", "SEP", "OCT", "NOV", "DEC"};  
int8\_t mouth=0,add=0; // 接收到的月份和要增加的月份
    \end{lstlisting}

        {\kaishu 在UART0中断服务函数中,以非阻塞方式接收字符,
        并进行判断,以确定对应的月份和月份增加数。}

        \begin{lstlisting}[language=C++]
void UART0_Handler() {
    .....
    UARTStringGetNonBlocking(msg); // 接收字符串
    for (int i=0;i<12;i++) {
        if (strncmp(msg, month[i], 3) == 0) {
            mouth = i;
            break;
        }
        if (i==11) {
            UARTStringPutNonBlocking("Invalid month!\n");
            return;
        }
    }
    
    // 确定增加的月份
    if (msg[3] == '+') { 
        add = (msg[4]-'0')*10 + (msg[5]-'0');
        mouth += add;
        mouth %= 12; // 防止超界
    } else if (msg[3] == '-') {
        add = (msg[4]-'0')*10 + (msg[5]-'0');
        mouth -= add;
        mouth = (mouth+12)%12; // 防止超界
    } else {
        UARTStringPutNonBlocking("Invalid command!\n");
        return;
    }

    UARTStringPutNonBlocking(month[mouth]); // 输出处理好的月份


}

    \end{lstlisting}
    
    \item 请根据上位机的命令,如“14:12+05:06”,字符串中:
    
    14:12 为分钟与秒,共 5 位,包括一个“:”。

    “+” 表示加运算符, “-” 表示减运算符,均为 1 位。

    05:06 为分钟与秒的变化量,共 5 位。包括一个“:”,范围 00:00~23:59。

    以上均为 ASCII 码,14:12+05:06 回之以 19:18。

    {\kaishu 在UART0中断服务函数中,对接收的字符串判断,并根据不同情况进行处理。
    根据输入字符串得到分钟和秒,然后根据加减号和变化量进行处理,最后输出处理好的分钟和秒。}

    \begin{lstlisting}
void UART0_Handler() {
    .....
    UARTStringGetNonBlocking(msg); // 接收字符串
    uint8_t min = (msg[0]-'0')*10 + (msg[1]-'0');
    uint8_t sec = (msg[3]-'0')*10 + (msg[4]-'0');
    uint8_t add_min = (msg[6]-'0')*10 + (msg[7]-'0');
    uint8_t add_sec = (msg[9]-'0')*10 + (msg[10]-'0');

    if (msg[5] == '+') {
        min += add_min;
        sec += add_sec;
        if (sec>=60) {    // 处理进位
            min += sec/60;
            sec %= 60;
        }
        if (min>=60) {    // 处理进位
            min %= 60;
        }
    } else if (msg[5] == '-') {
        min -= add_min;
        sec -= add_sec;
        if (sec<0) {    // 处理借位
            min -= 1;
            sec += 60;
        }
        if (min<0) {    
            min = 0;
            UARTStringPutNonBlocking("Invalid input!\n");
        }
    } else {
        UARTStringPutNonBlocking("Invalid command!\n");
        return;
    }
    char time_str[6]; // 创建一个字符数组来存储时间字符串
    sprintf(time_str, "%02d:%02d", min, sec); // 格式化时间
    UARTStringPutNonBlocking(time_str); // 输出时间 
}
    \end{lstlisting}
        


\end{enumerate}
        
\end{document}

